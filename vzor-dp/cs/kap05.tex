\chapter{Predikcia sekundárnej štruktúry}

Ako prvé opatrenie, na zlepšenie rýchlosti a presnosti predikovania nekonzerovvaných úsekov sme navrhli najprv riešiť jednoduchší problém a to predikciu sekundárnej štruktúry targetu a pomocou nej neskôr predikovať terciárnu štruktúru nekonzervovaných úsekov, vďaka čomu predpokladáme, že by sa mohol výrazne zmenšiť prehľadávaný priestor pri generovaní kandidátskych štruktúr algoritmom FARFAR. Principiálne teda najprv komparatívnym modelovaním napredikujeme sekundárnu target štruktúru a tú potom poskytneme ako vstup algoritmu FARFAR, ktorý ju použije pri predikcii terciárnej target štruktúry. Tento postup a jeho výsledky sme popísali aj v článku \cite{8218009}.

\section{Príprava dát}
Pretože algoritmus doplníme o komparatívne predikovanie sekundárnej štruktúry, bude algoritmus okrem sekvencie target molekuly a sekvencie aj terciárnej štruktúry template molekuly potrebovať na vstupe aj sekundárnu štruktúru tmeplate molekuly. Štruktúru očakávame v dot-bracket notácii. Sekundárnu štruktúru by bolo možné získať z terciárnej aj počas predikcie, ale vzhľadom na to, že by si užívateľ chcel dodať inú sekundárnu štruktúru sme sa rozholi, že sekundárna štruktúra bude dodaná ako ďalší vstupný parameter.


\indent Získať dáta sme sa rozhodli ich určením z terciárnej štruktúry programom DSSR (Defining the Secondary Structures of RNA) so software balíku x3DNA \cite{x3dna}. Na hromadné extrahovanie štruktúr sme si potom vytvorili dva krátke scripty, ktoré sa nachádzajú v \url{MyTools/PredictSecondaryStructures/}. DSSR vytvára v sekundárnej štruktúre base-pairs vyznačené jednoduchými zátvorkami, ako aj pseudo uzly vyznačené hránatými zátvorkami \ref{obr5.0}. Z dôvodu implementačných komplikácií a zladenia s algoritmom FARFAR sme sa rozhodli pracovať iba so sekundárnou štruktúrou označujúcou base-pairs a  preto pseudouzly neberieme v úvahu.
\begin{figure}%[p]\centering
\includegraphics[width=\textwidth]{../img/dssr}
\caption{Príklad súboru sekundárnej štruktúry získanej progrmaom DSSR zo štruktúry terciárnej. V prvom riadku sú príslušné typu nukleotudov a v druhom riadku je samotná sekundárna štruktúra v  dot-bracket reprezentácií.}
\label{obr5.0}
\end{figure}


\section{Predikcia sekundárnej štruktúry}
Sekundárnu štruktúru predikujeme rovnakým spôsobom, ako štruktúru terciárnu, teda komparatívnym modelovaním. Ako vstup potrebuje algoritmus dostať sekvenciu target molekuly a sekvenciu aj sekundárnu štruktúru template molekuly. 


\indent  Používame rovnakú target štruktúru ako pre terciárnu predikciu, čo znamená, že po ošetrenie medzier v zarovnaní \ref{3-indels}, teda vlastne po získanie konzervovaných úsekov v zarovnaní. Následne potrebujeme na zarovnanie namapovať sekundárnu template štruktúru. Tu musíme riešiť situáciu, že ak práve jeden nukleotid z base-pair nie je konzervovaný a druhý je, musíme v sekundárnej štruktúre preznačiť oba nukleotidy ako nespárované. Tiež to môžeme chápať tak, že sekundárnu štruktúru udržujeme stále správne uzátvorkovanú, čo znamená, že zátvorky rovnakého typu sa navzájom nekrížia a máme rovnaký počet pravých aj ľavých zátvoriek v sekundárnej štruktúre.


\indent Namapovaním sekundárnej štruktúry na alignment získame neúplnú sekundárnu štruktúru target molekuly. Túto následne predáme spolu s target sekvenciou nástroju RNAfold \cite{RNAfold}, ktorý predikuje sekundárnu štruktúru vzhľadom na obmedzenia plynúce z dodanej sekundárnej štruktúry. Dĺžka predikcie sa pohybuje v rádoch jednotiek až desiatok sekúnd, takže to nepredstavuje židne citeľné spomalenie celého algoritmu. 


\section{Integrácia do existujúceho algoritmu}
Do konfiguračného súboru sme pridali nastavenie, ktoré určuje, či algoritmus pracuje aj so sekundárnou štruktúrou, alebo nie.


\indent Z pohľadu začlenenia do algoritmu sa sekundárna štruktúra predikuje po kroku \ref{3-map}, ktorý mapuje terciárnu template štruktúru na alignment. V ďalšom kroku, ktorý ošetruje dlhé medzery už namapovanú sekundárnu štruktúru používame na to, aby sme k fragmentom terciárnej štruktúry slúžiacej ako template pre FARFAR vybrali aj fragmenty sekundárnej štruktúry. Vzhľadom na to, že FARFAR-u musíme dať validnú sekundárnu štruktúru v dot-bracket notácii v prípade, ak je do predikcie dlhého nekonzervovaného úseku vybraný z base-paru v sekundárnej štruktúre práve jeden nukleotid, musíme pridať aj druhý, aby sme zachovali validitu sekundárnej štruktúry. To isté musíme riešiť v ďalšom kroku, keď pripravujeme predikciu zvyšku štruktúry už bez dlhých nekonzervovaných úsekov. 


\indent Nakoniec bolo treba upraviť shell scripty tak, aby nakopírovali na správne miesta súbory so sekundárnymi štruktúrami a v parametri predať príslušný template súbor so sekundárnou štruktúrou algoritmu FARFAR.

\section{Experiment a Výsledky}
Za účelom overenia, vplyvu pridania predikcie sekundárnej štruktury do pôvodného algoritmu, sme urobili rovnaké testovanie nad rovnakými dátami, ako pri porovnávaní pôvodného algoritmu s ModeRNA. Skúšali sme napredikovať všetky páry s podobnosťou medzi 60\% - 90\% a dĺžkou 50-100 a 101-500 nukleotidov najprv pôvodnou verziou algoritmu Trooper, verziou algoritmu vylepšenou o predikciu sekundárnej štruktúry a referenčnou predikciou ModeRNA. 


\indent  Výsledky a porovnanie predikcií uvádzame v tabuľke \ref{tab5.1}. Z výsledkov vidíme, že pridanie predikcie sekundárnej štruktúry spôsobilo, že priemerná RMSD pre štruktúry s dĺžkou medzi 50-100 nukleotidov a 413 úspešne napredikovaných pároch sa zhoršila z 5,80Å na 8,23Å, čo približne zodpovedá výsledku 8,53Å, ktoré na predikcii rovnakých štruktúr dosiahla ModeRNA. Pri predikcií štruktúr dĺžky 101-500 nukleotidov a 98 úspešne napredikovaných pároch sa naopak priemerné výsledky predikcie s pridaním predikcie sekundárnej štruktýry výrazne zlepšili a to z 6,91Å na 3,46Å, čo je mierne lepšie ako výsledky ModeRNA, ktorá pri rovnakých podmienkach dosiahla priemer 3,72Å. 
\begin{table}[b!]
\centering
\begin{tabular}{ccccc}
\toprule
Veľkosť & Počet párov & Trooper bez SS & Trooper s SS & ModeRNA\\
\midrule
50-100  & 413 & 5,80  & 8,23 & 8,53\\
101-500  & 98 & 6,91  & 3,46 & 3,72\\
\bottomrule
50-500  &  511 & 5,95  & 7,31 & 7,61\\
\end{tabular}
\caption{Porovnanie priemernej RMSD predikcie RNA nástrojmi Trooper, Trooper s predikciou sekundárnej štruktúry a ModeRNA. }\label{tab5.1}
\end{table}


\indent Z výsledkov je teda zrejmé, že pridanie predikcie sekundárnej štruktúry a jej poskytnutie algoritmu FARFAR v niektorých prípadoch výsledky zlepšilo a v niektorých zhoršilo.Prečo boli zlepšené práve dlhšie štruktúry by mohlo byť vysvetliteľné tým, že môžu obsahovať dlhé nekonzervované úseky, ktorých predikcia je pre FARFAR náročnejšia. Príklad z obrázka \ref{obr5.1} práve podporuje takúto teóriu, kedy vidíme, že pôvodná predikcia FARFAR bez sekundárnej štruktúry napredikovala nekonzervovaný úsek zle, ale po pridaní sekundárnej štruktúry sa FARFAR-u podarilo úsek napredikovať oveľa presnejšie.
\begin{figure}%[p]\centering
\includegraphics[width=\textwidth]{../img/struct1}
\caption{Modrá štruktúra je experimentálne získana štruktúra molekuly 3DIG:X s dĺžkou 175 nukleotidov. Červená štruktúra je predikovaná algoritmom Trooper bez použitia sekundárnej štruktúry s výslednou RMSD na úrovni 14,44Å. Zelená štruktúra je predikcia urobená algoritmom Trooper s výslednou RMSD 4,44Å. PRe obe predikcie, bol použitý rovnaký template a to štruktúra 3DOU:A.}
\label{obr5.1}
\end{figure}


\indent Ďalej sme ešte skúmali značné zhoršenie výsledkov v triede štruktúr veľkostí 50-100 nukleotidov z RMSD 5,8Å na 8,23Å. Zastávame hypotézu, že by to mohlo byť spôsobené nesprávnou sekundárnou štruktúrou donanou algoritmu FARFAR, pretože inak boli podmienky oboch predikcií rovnaké a teda by sa priemerné výsledky nemali  od seba značne líšiť.


\indent Správnosť predikovanej sekundárnej štruktúry vieme popísať a klasifikovať štyrma mierami: true positives (správne určený existujúci base-pair), true negatives (správne určený nespárovaný nukleotid), false positives (nesprávne určený base-pair) a nakoniec false negative (v štruktúre by mal existovať base-pair, ale nie je napredikovaný). Z týchto štyroch ukazateľov sú prvé dva pozitívne a druhé dva negatívne. Pritom ale FARFAR-u môže uškodiť iba prípad false negative, pretože ten mu indikuje, že má v predikcii spárovať (a teda umiestniť blízko seba) dva nukleotidy. Prípady s false negatives by predikciu FARFAR-u nemal zlepšiť, ale ani zhoršiť, pretože nekladú žiadnu podmienku na spárovanie nukleotidov. 


\indent Zamerali sme sa predto na prípady false positives v predikovaných sekundárnych štruktúrach a analyzovali sme koreláciu medzi zhoršením, prípadne zlepšením predikcie algoritmu Trooper bez sekundárnej štruktúry a algoritmu Trooper s predikciou sekundárnej štruktúry vzhľadom na počet false positives v napredikovanej sekundárnej štruktúre určenej dodanej ako vstup algoritmu FARFAR. Výsledok je graficky znázornený na obrázku \ref{obr5.2} a podporuje našu hypotézu o tom, že čím viac vzrastal počet false positives v predikovanej sekundárnej štruktúre, tým bolo pravdepodobnejšie, že predikcia so sekundárnou štruktúrou sa oproti predikcii bez sekundárnej štruktúry zhorší. Preto si myslíme, že zhoršenie v niektorých preikciách bolo spôsobené hlavne zle napredikovanou sekundárnou štruktúrou.
\begin{figure}%[p]\centering
\includegraphics[width=\textwidth]{../img/corelation}
\caption{Závislosť medzi rozdielom výsledkov predikcie s a bez sekundárnej štruktúry a počtom false positive base-pairs v napredikovanej sekundárnej štruktúre. Červené body sú jednotlivé záznamy a čiarkovanou čiarou je zobrazená lineárna regresia týchto bodov.}
\label{obr5.1}
\end{figure}