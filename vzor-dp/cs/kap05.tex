\chapter{Predikcia sekundárnej štruktúry}

Ako prvé opatrenie, na zlepšenie rýchlosti a presnosti predikovania nekonzerovvaných úsekov sme navrhli najprv riešiť jednoduchší problém a to predikciu sekundárnej štruktúry targetu a pomocou nej neskôr predikovať terciárnu štruktúru nekonzervovaných úsekov, vďaka čomu predpokladáme, že by sa mohol výrazne zmenšiť prehľadávaný priestor pri generovaní kandidátskych štruktúr algoritmom FARFAR. Principiálne teda najprv komparatívnym modelovaním napredikujeme sekundárnu target štruktúru a tú potom poskytneme ako vstup algoritmu FARFAR, ktorý ju použije pri predikcii terciárnej target štruktúry. Tento postup a jeho výsledky sme popísali aj v článku \cite{8218009}.

\section{Príprava dát}



\section{Predikcia sekundárnej štruktúry}


\section{Integrácia do existujúceho algoritmu}


\section{Experiment}


\section{Výsledky}