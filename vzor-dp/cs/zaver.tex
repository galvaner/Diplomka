\chapter*{Závěr}
\addcontentsline{toc}{chapter}{Závěr}
Cieľom tejto práce bolo hlavne rozšíriť pôvodný algoritmus predikcie terciárnej štruktúry za pomoci vzorovej štruktúry o dva moduly, ktoré by zlepšili problematickú predikciu štruktúr obsahujúcich dlhé nekonzervované úseky. Okrem toho sme chceli predikciu zautomatizovať tak, aby nevyžadovala manuálne zásahy a následne naše výsledky porovnať s algoritmom ModeRNA.

\indent Prvý modul, ktorý vkladá do predikcie terciátnej štruktúry medzikrok predikcie sekundárnej štruktúry sme úspešne navrhli, implementovali a otestovali. Z testovania vyplynulo, že časť predikovaných štruktúr sa zlepšila a časť naopak zhoršila, za čo, ako sme analýzou ukázali, môžu hlavne nepresne napredikované sekundárne štruktúry target molekuly.
Návrh druhého modulu, pomocou ktorého vieme pri predikcií jednej target štruktúry použiť  viacero template štruktúr sa ukázal byť zložitejší. Prvé dva návrhy algoritmu sa pri testovaní ukázali ako nepoužiteľné, ale tretiu verziu algoritmu sa nám podarila naimplementovať a odladiť do takej miery, že sme boli schopní získať výsledky indikujúce, že pri nájdení správnej sekundárnej template štruktúry sme schopní výsledok predikcie zlepšiť, ale naopak pri použítí nevhodnej štruktúry sa môžu výsledky výrazne zhoršiť.
Implementáciu algoritmu sa nám podarilo úspešne automatizovať do takej miery, že na hromadnú predikciu štruktúr a vyhodnotenie výsledkov stačí spustiť štyri shell skripty. Vďaka tomu, sme dokázali výsledky algoritmu Trooper porovnať s výsledkami algoritmu ModeRNA, pričom sa ukázalo, že náš algoritmus bol vo verzii so sekundárnou štruktúrou v celkovom priemere RMSD o niečo presnejší.  

  
\indent Výsledky našej práce ukázali, že obe metódy majú potenciál, aby pri použití správnej sekundárnej štruktúry, alebo vhodnej  ďalšej template štruktúy dokázali  výsledky predikcie zlepšiť. Okrem toho sme ukázali, že naše aktuálne výsledky sú porovnateľné s konkurenčným algoritmom ModeRNA.
V budúcnosti by sme sa práve chceli zamerať na zlepšenie predikcie sekunárnych štruktúr a ďalšie vyladenie hľadania vhodných template štruktúr. Takisto by sme chceli zbaviť nášu implementáciu závislosti na algoritme FARFAR a predikciu nekonzervovaných úsekov zabezpečovať vlastným riešením.